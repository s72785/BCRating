
Einfache Wahrheiten für Geschäftsideen werden durch die Dezentralisierung in Frage gestellt.
Deshalb stellen Geschäftsinteressen auch zuerst die Notwendigkeit der Technologie zur Disposition.
Für das Ziel der Bewertungsmöglichkeit in Unternehmen wurde die Handhabbarkeit verschiedener Darstellungen betrachtet.
Dabei ergab sich das Netzdiagramm für vorausgewählten Kriterien als einfachste Möglichkeit um eine entscheidungsfähige Aussage zu treffen.
Ein mit linearem Aufwand berechenbares Entscheidungskriteriums wurde dazu vorgeschlagen.
Daran anschließend wäre die Entwicklung eines Werkzeugs denkbar, das die Bewertung auch im Gespräch mit Entscheidern ermöglicht.

Für das weitere Feld der \gls{DLT} sind zeitnah ähnliche Bewertungen möglich sobald genügend Informationen zur Verfügung stehen.
Für Unternehmen ist hier eine gesteigerte Attraktivität beim Verzicht auf die Blockstruktur zu erwarten sobald das Problem der Zentralisierung dort behoben ist.
Insgesamt steht die Technologie noch am Anfang und wesentliche Weiterentwicklungen werden erwartet.
Die Performance wäre eine gesonderte Untersuchung wert, wobei ein Schwerpunkt auf der Vergleichbarkeit der Bedingungen liegen sollte.

Es ist nicht zu erwarten, dass \gls{BCT} zeitnah Erfolge für eine bessere Auditierbarkeit der Buchhaltung feiert.
%~ Aber für die Verbesserung des \gls{GPM} an kritischen Stellen zwischen Geschäftspartnern und eventuell auch zur Kostenminderung bei \zB{} vorgeschriebenen Auskunftsansprüchen zum Verbraucher im Rahmen verstärkter Automatisierung ist das Interesse hoch.
Aus der Betrachtung der industriell geförderten Kandidaten, ergibt einen geringeren Schwerpunkt Wert auf Erlaubnisfreiheit bzw. eine möglichst gute Sicherheit (\gls{BFT}),
denn vielmehr auf geringe Betriebskosten und Kontrolle über die gespeicherten Daten.
Der Interessenkonflikt zwischen offenen Standards und dem Wissensvorsprung für die Beteiligten von Konsortien wird ebenfalls bestehen bleiben.
Hier sind aber gerade am Projekt Hyperledger große Fortschritte erkennbar.
%~ Mehr Details werden nach und nach publiziert werden und der Kreis der Adressaten im Sinne der Industrie erweitert werden.
Für eine bessere Vergleichbarkeit ist naheliegendes Ziel für weitere Betrachtungen \ua{} die konkrete Bemessung der \gls{BFT} für verfügbaren Konsensalgorithmen.
%~ Diese Betrachtung wird durch die Beteiligung von Forschungseinrichtungen auch immer wieder aufgenommen.
%~ und frei verfügbaren \gls{BCI}

%~ Das entworfene 

%~ Grundlegende Eigenschaften wie die Dezentralisierung wiedersprechen dem Gewinnstreben indem eine weitgehende Dezentralisierung in Verbindung mit dem Grenzkostenprinzip droht die Marge zu schmälern.
%~ Deshalb kommt es darauf an, ein Geschäftsmodell zum Endkunden zu entwerfen das von einem Miteinander lebt oder sich auf Anwendungsfälle zur Effizienzsteigerung in der eigenen Firma oder zu Geschäftspartner zu fokusieren.

