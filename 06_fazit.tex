
Für das Ziel der Bewertungsmöglichkeit in Unternehmen wurde die Handhabbarkeit verschiedener Darstellungen betrachtet.
Dabei ergab sich das Netzdiagramm als einfachste Möglichkeit ein entscheidungsunterstützendes Bild zu liefern.
Dazu wurde ein mit linearem Aufwand berechenbares Entscheidungskriterium vorgeschlagen.
Naheliegend wäre die Entwicklung eines Werkzeugs, das die Bewertung auch im Gespräch mit Entscheidern ermöglicht
und Daten zur Beurteilung weiterer Kandidaten sammelt.
Ein weiteres Ziel für weitere Betrachtungen ist bessere Vergleichbarkeit \ua{} für die konkrete Bemessung der Fehlertoleranz verfügbarer Konsensalgorithmen.

Einfache Wahrheiten für Geschäftsideen werden durch die Dezentralisierung in Frage gestellt.
Deshalb stellen lange erfolgreiche Strategien zur Abschottung gegen Mitbewerber auch zuerst die weitreichenden Möglichkeiten der Technologie zur Disposition.
Es ist nicht zu erwarten, dass Blockchain Technologie zeitnah Erfolge für eine bessere Auditierbarkeit der Buchhaltung feiert.
%~ Aber für die Verbesserung des \gls{GPM} an kritischen Stellen zwischen Geschäftspartnern und eventuell auch zur Kostenminderung bei \zB{} vorgeschriebenen Auskunftsansprüchen zum Verbraucher im Rahmen verstärkter Automatisierung ist das Interesse hoch.
Aus Sicht der industriellem Förderer der Blockchain-Implementierungen ergibt sich kein Schwerpunkt auf Erlaubnisfreiheit oder eine möglichst hohe Fehlertoleranz des Konsens.
Stattdessen liegen die Prioritäten auf Speicherkapazität, Transaktionsgeschwindigkeit sowie geringe Betriebskosten und Kontrolle über die gespeicherten Daten.
Der Interessenkonflikt zwischen offenen Standards und dem Wissensvorsprung für die Beteiligten von Konsortien wird daher mittelfristig bestehen bleiben.
Beim Projekt Hyperledger sind erste Fortschritte bei der Einbindung von Forschungseinrichtungen erkennbar.
%~ Mehr Details werden nach und nach publiziert werden und der Kreis der Adressaten im Sinne der Industrie erweitert werden.

Für das erweiterte Feld der Distributed Ledger Technologie sind zeitnah ähnliche Bewertungen für Vertreter wie IOTA und Hashgraph zu erwarten.
Es kann eine gesteigerte Attraktivität durch bessere Performance beim Verzicht auf die Blockstruktur erwartet werden.
Insgesamt steht die Technologie noch am Anfang und wesentliche Weiterentwicklungen zur Dezentralisierung stehen noch aus.
%~ Die Performance wäre eine gesonderte Untersuchung wert, wobei ein Schwerpunkt auf der Vergleichbarkeit der Bedingungen liegen sollte.

%~ Diese Betrachtung wird durch die Beteiligung von Forschungseinrichtungen auch immer wieder aufgenommen.
%~ und frei verfügbaren \gls{BCI}

%~ Das entworfene 

%~ Grundlegende Eigenschaften wie die Dezentralisierung wiedersprechen dem Gewinnstreben indem eine weitgehende Dezentralisierung in Verbindung mit dem Grenzkostenprinzip droht die Marge zu schmälern.
%~ Deshalb kommt es darauf an, ein Geschäftsmodell zum Endkunden zu entwerfen das von einem Miteinander lebt oder sich auf Anwendungsfälle zur Effizienzsteigerung in der eigenen Firma oder zu Geschäftspartner zu fokusieren.

