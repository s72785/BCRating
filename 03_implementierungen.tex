%Wie unter \ref{}

%davon abweichend aus den erkannten Nachteilen heraus nach Blockchains gesucht werden, die dahingehend einen anderen Weg -- ohne \gls{PoW}, mit höherer Transaktionsgeschwindigkeit und mit Rücksicht auf den Datenschutz -- beschreiten.

Wenn eine neue \gls{BC} auftaucht, können Beobachter sehr unterschiedliche darauf reagieren.
Ein häufiger Schwerpunkt liegt bei der nativen Währung -- die Betrachtung als Investitionsobjekt.
Der nächste sucht den primären Anwendungsfall und hat eher einen Fokus auf die praktische Nutzbarkeit.
%Aber abseits des Kryptowährungsmarktes weitere Blockchain-Implementierung aufzufinden ist nicht ganz so einfach.
An dieser Stelle soll die davon abweichend die Hypothese~\rom{2} verfolgt werden.
Wenn Unternehmen bereits aktiv mit \gls{BCI} arbeiten, sollten Beispiele auffindbar sein.

Die in der \nameref{einleitung} bereits erwähnte \gls{HSM} bot im September~2017 eine Woche \enquote{Autumn School} zum Thema Blockchain an.\footnote{Programm und Informationen \autocite{w:hsmw-bccm-as}}
Der Schwerpunkt wurde wahrscheinlich durch die in der Nähe befindlichen und bei der Gründung von Ethereum beteiligte Firma \mbox{slock.it} beeinflusst.
In der Retrospektive ist interessant, welche konkreten \gls{BCI} -- außer Ethereum oder Bitcoin -- Firmen betrachten oder im Einsatz haben.
Hier sticht der Name \emph{Deutschen Bundesbank} mit der Vorführung eines Prototyps hervor.
Dieser wurde mit \emph{Quorum} -- einem Fork von Ethereum -- umgesetzt. Dies scheint also ein beachtenswerter Vergleichskandidat zu sein.
Da das Projekt von der \mbox{Linux Foundation} mit Unterstützung zahlreicher namhafter Firmen, u.a Intel, SAP, Daimler, IBM und Accenture, gehostet wird ist davon auszugehen, dass sich die Investition für beteiligte Unternehmen lohnt.
Weiterhin hat die Firma \emph{Sopra Steria} die Umsetzung von Kundenprojekten mittels \emph{BigchainDB} und den Einsatz von \emph{Hyperledger} durch Kunden erwähnt.
Die hinter \emph{BigchainDB} stehende gleichnamige Firma ist ebenfalls Mitglied im \emph{Bundesverband Blockchain} und arbeitet mit der \gls{IPDB} deren Finanzierung durch eine Stiftung\footnote{Blogpost zur Gründung \autocite{w:ipdb-foundation}} sichergestellt ist.
Damit hätten wir einen weiteren bereits bei Firmenkunden eingesetzten Kandidaten.

%~ Ein internationaler Konkurrent aus China ist \emph{NEO} -- vormals \gls{ANS}.
%~ Die Dokumentation  überladene Design der Website gepaart mit bereits im Whitepaper fehlerhafter englischer Übersetzung\footnote{\enquote{NEO White Paper} \autocite{w:neo-whitepaper}} machen wenig Freude beim Lesen.
%~ Beworben https://steemit.com/cryptocurrency/@basiccrypto/almost-everything-you-wanted-to-know-about-neo-part-1-of-2

Auf der Suche nach einem weiteren Kandidaten leuchtet die Schlagzeile\footnote{\autocite{w:rebealley:sydney}} über einen Geschwindigkeitsrekord der \emph{Red Belly Blockchain}\footnote{\autocite{p:rbbc}} auf.
Über diese Schlagzeilen und ein Whitepaper hinaus finden sich aber kaum Informationen, weshalb die bisher gefundenen drei \gls{BCI} genügen sollen.
%~ Das Projekt \emph{Hyperledger} der \emph{Linux Foundation} bietet dagegen neben Dokumentation auf GitHub\footnote{Hyperledger \cite{w:github-hyperledger}} zahlreiche Mitglieder aus der Industrie, darunter auch die Deutsche Börse und die Firma SAP.

\section{BigchainDB}

Der Name suggeriert bereits, dass es sich bei \emph{BigchainDB}\footnote{Whitepaper \autocite{p:bigchaindb}} primär um die Nutzbarmachung der \gls{BCT} für Datenbanken handelt.
%~ Tatsächlich wird auch die Blockchain in einer Datenbank gehalten.
Bisher wurden Treiber für RethinkDB und für MongoDB -- zwei Vertreter von noSQL-Datenbanken -- implementiert. Die Autoren konzentrieren sich nach eigener Aussage auf die Skalierbarkeit bei Erhaltung der Dezentralisierung und Unveränderbarkeit sowohl für eine globale öffentliche \gls{BC} auch auch private \gls{BC}.
Nicht ganz intuitiv ob des Namens ist die klare Vorgabe\footnote{\autocite{w:bcdb:features}}: Statt der Daten werden nur Hash und Zugangstoken in der Blockchain gespeichert.

Für verbesserten Datenschutz (engl. privacy) von Daten sollen \enquote{Privacy Protocols} sorgen.\footnote{\cite{w:github-bigchaindb-pp}}
Dabei sollen Authentifizierung und Beweise -- für u.a. Sharing, Storage und Replikation -- die einzelnen Operationen zusichern.
Für die redundante Ablage ist auch die Nutzung von \gls{IPFS} oder vergleichbaren verteilten Dateisystemen und Datenbanken angedacht.
Zusätzlich wurden Vermögenswerte (engl. Assets) nativ ermöglicht.
Diese sollen auch zur Authentifizierung und Rechteverteilung dienen.
Die Verbindung verschiedener \gls{BC} soll z.B. über das Interledger Protokoll\footnote{A Protocol for Interledger Payments \autocite{p:interledger}}\label{first:interledger} ermöglicht werden.

%~ Die Roadmap\footnote{\url{https://github.com/bigchaindb/org/blob/master/ROADMAP.md}} hat noch viele Features für die Zukunft
Entwicklungswerkzeuge sind ein wesentliche Betätigungsfeld bei BigchainDB.
Bisher stehen Treiber für Python und JavaScript (inkl. NodeJS) zur Verfügung.
Änderungen an der API werden für Nutzer nachvollziehbaren Versionsschritten gemacht.
Das Deployment von Nodes mittels Microsoft Azure und Kubernetes ist seitens des Kernteams dokumentiert.\footnote{\cite{w:bigchaindb-proddepl}}

Bisher ist BigchainDB nicht zuverlässig dezentral einsetzbar und nicht \gls{BFT}.
Teilnehmer müssen sich für den Abgleich mit der öffentlichen, globalen \emph{IPDB} mit dem zentralen Datenbank-Cluster verbinden.
Es ist kein Anreizsystem erkennbar, dass für einen Betrieb des Netzwerkes sorgen würde.
Auch der Konsensmechanismus ist noch fest verankert und soll für private Instanzen erst noch austauschbar gestaltet werden.\footnote{\url{https://www.reddit.com/r/Bitcoin/comments/4j7wjf/bigchaindb_a_prime_example_of_blockchain_bullshit/}}

Das Angebot an Firmen reicht von Workshops über Hilfe bei der Entwicklung von Anwendungsfällen bis hin zu Support.
Die Unterhaltung von Datenbankinstanzen gehört auch zum Geschäftsmodell; die Preise sind nicht transparent.\footnote{\cite{w:bigchaindb-enterprise}}
%~ * federation with pki => https://docs.bigchaindb.com/projects/server/en/latest/production-deployment-template/index.html
%~ * tbd: labs.ipdb.com

%~ * no smart contracts? (\in electronic contracting)

%~ https://www.bigchaindb.com/
%~ https://github.com/bigchaindb/bigchaindb
%~ https://readthedocs.org/projects/bigchaindb/
%~ https://docs.bigchaindb.com/en/latest/

%~ Skalierung durch Sharding

%~ Participants: 
  %~ --- Dimi: ascribe, with Bruce, building stuff on bitcoin, then BigchainDB
%~ -- Bitcoin amazing properties, but hard with storing stuff
  %~ --- Kamal: enterprise solutions, business development, customer, before software 
%~ - Hyperledger "is good to automate a business process"
%~ - Interledger --> connect different blockchains
%~ - Public Utility Networks --> consider rewards
%~ - Acyclic graph, a la bitcoin transaction
%~ - Scalability through sharding
%~ - single type of asset -- bitcoin in bitcoin
  %~ -- no minting of assets yourself
%~ - in BigchainDB mutliple types of assets
   %~ -- minting is fine
%~ - Federated Consensus Architecture
   %~ -- whitelist of federation nodes
%~ - replication protocols security
%~ - append-only database
%~ - deletion: in 1.0 Bigchain not there yet, maybe in 2.0
  %~ -- archiving may be possible
  %~ -- pure deletion --> not really (future-proof encryption protocols!) 
%~ - the other nodes will only accept valid transaction for replication (of for example deletion request, drop db, etc)
%~ - Consensus
%~ -- transactions validation
%~ -- fault tolerance
%~ - privacy
   %~ -- store only access control token to data in the blockchain (a pointer where the data is)
   %~ -- store only claims which contain enough information about the underlying data
%~ -  proof of replication, storage
%~ - privacy protocols from BigChainDB: 
%~ - IPDB -- BigChain DB in a federated setting
%~ - Federation = define voting nodes
   %~ -- Nodes authentication for voting
  %~ -- see production deployment template on Bigchain DB
%~ -  "Digital twins"
%~ - IRP/IMS --> supply chain management communication protocols
%~ - PharmaTrustChain
   %~ -- "Smart contract" --> very easily implemented 
   %~ -- Role-based access control,
%~ - Similar to MongoDB, GPL license --> code modification--> available to the community, 
   %~ -- enterprise version
%~ -- Monthly subsription
%~ - Transaction times@BigchainDB 
    %~ -- 2 thousand transaction per second 
    %~ -- a couple of seconds for transaction
%~ - Guarantees/consensus --> documentation 
   %~ -- plugable consensus


\section{Quorum}\label{impl:quorum}

Quorum ist ein Ethereum-Fork\footnote{Auf Basis des Go~Ethereum Clients \autocite{w:quorum-jpmorgan:github}} für den Unternehmenseinsatz und profitiert insbesondere von den Werkzeugen, die für Ethereum entwickelt werden.
Es gibt hierbei private Transaktionen, deren Inhalt nur den Beteiligten unter den Teilnehmern bekannt wird, und öffentliche Transaktionen, deren Inhalt für alle Teilnehmer zugänglich ist.
Der Konsens ist konfigurierbar% wird durch einen \gls{glos:SmartContract} gebildet
. Dies ermöglicht den späteren Einsatz eines anderen Konsensalgorithmus.
Im Weiteren ist der in Quorum~2.0 und von Ethereum abweichende Raft\footnote{The Raft Consensus Algorithm \autocite{p:raft}} kurz beschrieben.

% uorumChaim
% erlaubt ein Voting durch in der Rolle der \emph{Voter} befindliche Teilnehmer. Neben den Votern gibt es zwei weitere \emph{Rollen} -- für das Erstellen der Blöcke die \emph{Maker} (können Voter sein) und die lediglich validierenden \emph{Observer} (können keine weitere Rolle einnehmen). Die ersten beiden Rollentypen können einem Observer ihre eigene Rolle durch eine Transaktion verleihen.
%Um die Wahrscheinlichkeit einer gleichzeitigen Blockerstellung bei mehreren Makern zu verringern werden individuelle Wartezeiten rundenbasierend zufällig neu gesetzt; ein Maker beginnt erst mit der Erstellung eines Blockes wenn seine Wartezeit abgelaufen ist.
%https://github.com/jpmorganchase/quorum/wiki/QuorumChain-Consensus

Der \enquote{Raft-based consensus}\footnote{\url{https://github.com/coreos/etcd/tree/master/raft}} ist ein Alternativer Konsensalgorithmus für private \gls{BC}. Die Teilnehmer sind in die drei Rollen unterteilt:
\emph{Leader}, \emph{Follower} und \emph{Candidate}. Im Normalfall gibt es genau einen \emph{Leader}, der als Hub für alle Anfragen der \emph{Follower} zuständig ist.
Der \emph{Candidate} wird nur benötigt um einen neuen \emph{Leader} zu bestimmen.
Dafür wird eine Wahl per Losverfahren nach dem Abschluss einer Amtszeit (engl. term) des \emph{Leader} angesetzt.
Sofern eine einfache Mehrheit einen neuen \emph{Leader} gewält hat, beginnt eine neue Amtszeit.
Es werden sehr schnelle Transaktionszeiten (im Millisekundenbereich) ermöglicht und es entstehen keine Blöcke soweit keine Transaktionen vorhanden sind;
d.h. der Effekt von unnötige, leeren Blöcken auf privaten Blockchains tritt nicht auf.
Allerdings verzichten die Autoren nach eigener Aussage vollständig auf die Byzantine Fehlertoleranz zugunsten eines leicht verständlichen und in der Praxis einsetzbaren Design. 

Als Fork profitiert Quorum von der Community und Weiterentwicklungen aus dem Umfeld von Ethereum.
Eigene Dokumentation und Entwicklungsergebnisse um das Projekt sind nicht öffentlich auffindbar. 
Jedes der privaten Netzwerke kann neue Features übernehmen oder auf dem bewährten Stand bleiben
ohne dabei abhängig von der Exponierung einer globalen \gls{BC} zu werden.

\section{Hyperledger Fabric}\label{impl:hyperledger}

Das Projekt wurde Dezember 2015 von der Linux Foundation begonnen und hat nicht eine spezifische Blockchain sondern die \emph{gemeinsame Entwicklung} der Technologie in Form mehrerer Frameworks als \emph{Open~Source Software} zum Ziel.
Die \emph{Mitgliedschaft} im Projekt und damit Vorteile für den intransparenten gegenseitigen Austausch und Einflussnahme sind jedoch jedoch an Bedingungen, für gewinnorientierte Teilnehmer an jährliche Gebühren zur Finanzierung, gebunden.\footnote{\cite{w:hyperledger:membership}}
Im Weiteren geht es um das Framework \emph{HyperLedger~Fabric}, eine \gls{PBC} mit ausdrücklichem Fokus auf konsortiale Nutzung zwischen Geschäftspartnern.

In der bereits sehr umfangreichen \emph{öffentlich verfügbaren Dokumentation}\footnote{\cite{w:hyperledger:doc}} sind Begriffe nicht konsolidiert.
U.a. wird an einigen Stellen von Smart~Contracts gesprochen, an anderen von Chaincode.
Und auch \gls{DLT} und \gls{BC} werden gleichermaßen verwendet.
Details sind Nutzern außerhalb der beteiligten Firmen allein durch die Dokumentation schwer zugänglich.\footnote{\cite{p:hyperledger:arch}}
Dank der Arbeit der \gls{HAWG} \cite{p:hyperledger:consensus}, die zwar keine Fehlertoleranz in Zahlen, aber immerhin eine Bewertung der seit 2017 verwendeten Konsensmethode offenbart.
Der Ausbau der Zusammenarbeit mit der \emph{Community} ist geplant.

Der Standard \emph{Konsensmechanismus} \enquote{Kafka}\footnote{\cite{w:kafka}}, der bei \nameref{impl:hyperledger} Fabric Verwendung findet, wird durch eine vereinfachte Stimmabgabe (engl. Voting) der Teilnehmer (engl. Peers) in Verbindung mit \gls{glos:Chaincode} zur Überprüfung der Transaktion gestaltet; das Verfahren ist nicht \gls{BFT}.\footnote{\cite{w:hyperledger:txworkflow}}
Die Umsetzung als Protokoll für private Blockchains in Konsortien stellt auch darauf ab, dass Identitäten bekannt sind.
\emph{Erlaubnisfreiheit} ist damit nicht gegeben.
Der Vorteil ist allerdings die austauschbare Gestaltung des Konsensprotokolls und parallele Frameworks innerhalb des Hyperledger Projektes die Alternativen bereitstellen.
%~ Durch den Verzicht auf \gls{BFT} in Verbindung mit der vorab notwendigen Akkreditierung von Teilnehmern ist Transaktionsgeschwindigkeit und -Anzahl kein

Hyperledger erlaubt eine gute Integration in bestehende Datenbanksysteme und eignet sich insbesondere zur Automatisierung von Geschäftprozessen.
Damit kann der Einsatz von \gls{BCT} parallel und frühzeitig initiiert werden.

%~ * issues around centralization
%~ * good for automation of business processes
%~ * better in-stack
%~ * not as trusted data sources
%~ * interledger to be adopted (in bigchaindb)
