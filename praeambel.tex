\documentclass[german,12pt,paper=a4,DIV=calc,twoside, openright]{scrreprt}
%~ parskip=,%
%~ headsepline,
%~ footsepline=openright,
%~ parskip=twoside,
%~ openright,   % openleft, openright, openany

\usepackage[a-1b,latxmp]{pdfx}
\usepackage[T1]{fontenc}
\usepackage[utf8]{inputenc}
\usepackage[ngerman]{babel}

% Einzugsart
\setlength{\parindent}{0em} 
\setlength{\parskip}{2em}

% spacing for paragraphs but no indent
\usepackage[onehalfspacing]{setspace}

%****************
% define medatata
% found on: http://golatex.de/pdfx-bzw-pdf-a-oder-pdf-x-t4427.html
%________________
% pub
\newcommand{\projektname}{Systematische Bewertung von\\Blockchain-Implementierungen\\im Unternehmenskontext} %mit umbrüchen
\def\Title{Systematische Bewertung von Blockchain-Implementierungen im Unternehmenskontext}%Titel nach Absprache mir Betreuerin
\def\Subject{Abschlussarbeit zum Studium Wirtschaftsinformatik (Bachelor)}
\def\Keywords{Blockchain, Distributed Ledger Technology, BigchainDB, Quorum, Hyperledger, Bitcoin, Ethereum, Bewertung, Benchmarking}
%~ \def\Publisher{Hochschule für Technik und Wirtschaft Dresden (HTW Dresden)}
\def\Copyright{Creative Commons: Namensnennung und Weitergabe unter gleichen Bedingungen 4.0}
\def\CopyrightIMG{\includegraphics[width=2cm]{img/CC-BY-SA_icon}}
\def\CopyrightURL{https://creativecommons.org/licenses/by-sa/4.0/}

% priv
\InputIfFileExists{customize-priv}{}{
% if not
\def\Author{Autorennamen}
\def\AuthorID{00000}
\def\ReviewerA{Prof.~Dr. A}
\def\ReviewerB{Dipl.-Wirt.-Inf. B}
\def\SupervisorA{Dipl.-Inf. C}
\def\SupervisorB{}
\def\DocDate{\today}
\def\WritePlace{CityTownPlace}
\def\RepoURL{}
}
%\input{customize-priv}

\def\setRGBcolorprofile{sRGB_IEC61966-2-1_black_scaled.icc}
\def\setCMYKcolorprofile{coated_FOGRA39L_argl.icc}
%***************************************************************************
% \convertDate converts D:20080419103507+02'00' to 2008-04-19T10:35:07+02:00
% found on: http://support.river-valley.com/wik.....ompliant_PDFs_from_pdftex
%___________________________________________________________________________
\def\convertDate{%
    \getYear
}
{\catcode`\D=12
 \gdef\getYear D:#1#2#3#4{\edef\xYear{#1#2#3#4}\getMonth}
}
\def\getMonth#1#2{\edef\xMonth{#1#2}\getDay}
\def\getDay#1#2{\edef\xDay{#1#2}\getHour}
\def\getHour#1#2{\edef\xHour{#1#2}\getMin}
\def\getMin#1#2{\edef\xMin{#1#2}\getSec}
\def\getSec#1#2{\edef\xSec{#1#2}\getTZh}
\def\getTZh +#1#2{\edef\xTZh{#1#2}\getTZm}
\def\getTZm '#1#2'{%
    \edef\xTZm{#1#2}%
    \edef\convDate{\xYear-\xMonth-\xDay T\xHour:\xMin:\xSec+\xTZh:\xTZm}%
}
 
\expandafter\convertDate\pdfcreationdate

\pdfinfo{%
    /Title    (\Title)
    /Author   (\Author)
    /Subject  (\Subject)
    /Keywords (\Keywords)
    /ModDate  (\pdfcreationdate)
    /Trapped  /False
    %~ /Publisher(\Publisher)
	/Copyright(\Copyright)
	/CopyrightURL(\CopyrightURL)
	/setRGBcolorprofile(\setRGBcolorprofile)
	/setCMYKcolorprofile(\setCMYKcolorprofile)
}


%~ \usepackage{lmodern}
%~ \PrerenderUnicode{äöüÄÖÜß}
%~ \usepackage[protrusion=true,expansion=true]{microtype} % Better typography

% needs to be (like) arial - uggly
%~ \usepackage[scaled]{helvet}
%~ \renewcommand*{\familydefault}{\sfdefault}

\usepackage[autostyle=true,german=quotes]{csquotes}
\usepackage{graphicx}
\usepackage{tikz}
%~ % Skalieren auf breite oder Höhe
\usepackage{environ}

\makeatletter
\newsavebox{\measure@tikzpicture}
\NewEnviron{scaletikzpicturetowidth}[1]{%
  \def\tikz@width{#1}%
  \def\tikzscale{1}\begin{lrbox}{\measure@tikzpicture}%
  \BODY
  \end{lrbox}%
  \pgfmathparse{#1/\wd\measure@tikzpicture}%
  \edef\tikzscale{\pgfmathresult}%
  \BODY
}
\NewEnviron{scaletikzpicturetoheight}[1]{%
  \def\tikz@width{#1}%
  \def\tikzscale{1}\begin{lrbox}{\measure@tikzpicture}%
  \BODY
  \end{lrbox}%
  \pgfmathparse{#1/\ht\measure@tikzpicture}%
  \edef\tikzscale{\pgfmathresult}%
  \BODY
}
\makeatother
\newlength{\imgwidth}
\newlength{\imgwidthmax}
\setlength{\imgwidthmax}{0.5\paperheight}
\newcommand\scalegraphics[1]{%
    \settowidth{\imgwidth}{\includegraphics{#1}}%
    \setlength{\imgwidth}{\minof{\imgwidthmax}{\minof{\imgwidth}{\textwidth}}}%
    \includegraphics[width=0.9\imgwidth]{#1}%
}

\usepackage{color}
%~ \usepackage{pdfpages}
\usepackage{comment}

% roman numbers support
\makeatletter
\newcommand*{\rom}[1]{%#1%
\expandafter\@slowromancap\romannumeral #1@%
}
\makeatother


% spacing of paragraphs influences lists, so fix that here 
%~ \usepackage{enumitem}
%~ \setlist[itemize]{parsep=0pt}
%~ \setlist[enumerate]{parsep=0pt}

% if we need listings for code or cli
%~ \usepackage{listings}
%~ \makeatletter
%~ \lst@Key{matchrangestart}{f}{\lstKV@SetIf{#1}\lst@ifmatchrangestart}
%~ \def\lst@SkipToFirst{%
    %~ \lst@ifmatchrangestart\c@lstnumber=\numexpr-1+\lst@firstline\fi
    %~ \ifnum \lst@lineno<\lst@firstline
        %~ \def\lst@next{\lst@BeginDropInput\lst@Pmode
        %~ \lst@Let{13}\lst@MSkipToFirst
        %~ \lst@Let{10}\lst@MSkipToFirst}%
        %~ \expandafter\lst@next
    %~ \else
        %~ \expandafter\lst@BOLGobble
    %~ \fi}
%~ \makeatother
%~ \definecolor{mygreen}{rgb}{0,0.6,0}
%~ \definecolor{mygray}{rgb}{0.5,0.5,0.5}
%~ \definecolor{mymauve}{rgb}{0.58,0,0.82}
%~ \lstset{ %
  %~ backgroundcolor=\color{white},   % choose the background color; you must add \usepackage{color} or \usepackage{xcolor}; should come as last argument
  %~ basicstyle=\ttfamily\footnotesize,        % the size of the fonts that are used for the code
  %~ breakatwhitespace=false,         % sets if automatic breaks should only happen at whitespace
  %~ breaklines=true,                 % sets automatic line breaking
  %~ captionpos=b,                    % sets the caption-position to bottom
  %~ commentstyle=\color{mygreen},    % comment style
  %~ deletekeywords={...},            % if you want to delete keywords from the given language
  %~ escapeinside={\%*}{*)},          % if you want to add LaTeX within your code
  %~ extendedchars=true,              % lets you use non-ASCII characters; for 8-bits encodings only, does not work with UTF-8
  %~ frame=single,	                   % adds a frame around the code
  %~ keepspaces=true,                 % keeps spaces in text, useful for keeping indentation of code (possibly needs columns=flexible)
  %~ keywordstyle=\color{blue},       % keyword style
  %~ language=Octave,                 % the language of the code
  %~ morekeywords={*,...},            % if you want to add more keywords to the set
  %~ numbers=right,                   % where to put the line-numbers; possible values are (none, left, right)
  %~ numbersep=5pt,                   % how far the line-numbers are from the code
  %~ numberstyle=\tiny\color{mygray}, % the style that is used for the line-numbers
  %~ matchrangestart=t,			   % show actual range from file
  %~ rulecolor=\color{black},         % if not set, the frame-color may be changed on line-breaks within not-black text (e.g. comments (green here))
  %~ showspaces=false,                % show spaces everywhere adding particular underscores; it overrides 'showstringspaces'
  %~ showstringspaces=false,          % underline spaces within strings only
  %~ showtabs=false,                  % show tabs within strings adding particular underscores
  %stepnumber=2,                    % the step between two line-numbers. If it's 1, each line will be numbered
  %~ stringstyle=\color{mymauve},     % string literal style
  %~ tabsize=2,	                   % sets default tabsize to 2 spaces
  %~ title=\lstname                   % show the filename of files included with \lstinputlisting; also try caption instead of title
%~ }
%~ \lstset{literate=
  %~ {á}{{\'a}}1 {é}{{\'e}}1 {í}{{\'i}}1 {ó}{{\'o}}1 {ú}{{\'u}}1
  %~ {Á}{{\'A}}1 {É}{{\'E}}1 {Í}{{\'I}}1 {Ó}{{\'O}}1 {Ú}{{\'U}}1
  %~ {à}{{\`a}}1 {è}{{\`e}}1 {ì}{{\`i}}1 {ò}{{\`o}}1 {ù}{{\`u}}1
  %~ {À}{{\`A}}1 {È}{{\'E}}1 {Ì}{{\`I}}1 {Ò}{{\`O}}1 {Ù}{{\`U}}1
  %~ {ä}{{\"a}}1 {ë}{{\"e}}1 {ï}{{\"i}}1 {ö}{{\"o}}1 {ü}{{\"u}}1
  %~ {Ä}{{\"A}}1 {Ë}{{\"E}}1 {Ï}{{\"I}}1 {Ö}{{\"O}}1 {Ü}{{\"U}}1
  %~ {â}{{\^a}}1 {ê}{{\^e}}1 {î}{{\^i}}1 {ô}{{\^o}}1 {û}{{\^u}}1
  %~ {Â}{{\^A}}1 {Ê}{{\^E}}1 {Î}{{\^I}}1 {Ô}{{\^O}}1 {Û}{{\^U}}1
  %~ {œ}{{\oe}}1 {Œ}{{\OE}}1 {æ}{{\ae}}1 {Æ}{{\AE}}1 {ß}{{\ss}}1
  %~ {ű}{{\H{u}}}1 {Ű}{{\H{U}}}1 {ő}{{\H{o}}}1 {Ő}{{\H{O}}}1
  %~ {ç}{{\c c}}1 {Ç}{{\c C}}1 {ø}{{\o}}1 {å}{{\r a}}1 {Å}{{\r A}}1
  %~ {€}{{\euro}}1 {£}{{\pounds}}1 {«}{{\guillemotleft}}1
  %~ {»}{{\guillemotright}}1 {ñ}{{\~n}}1 {Ñ}{{\~N}}1 {¿}{{?`}}1
  %~ {…}{{\ldots}}1 {–}{{--}}1
%~ }



%~ \usepackage[authoryear]{natbib}
% so:
\usepackage[backend=biber,%
citestyle=authoryear,%
urldate=comp,dateabbrev=false% for online sources date formatting
]{biblatex}
\addbibresource{quellen.bib}
\renewcommand{\bibfont}{\small}

% define string for bibliorphy urldate
\DefineBibliographyStrings{english}{%
urlseen = {Retrieved},
}
\DefineBibliographyStrings{german}{%
urlseen = {Stand/Abgerufen},
}

% Glossar
\usepackage[ngerman]{translator}
%Paket laden
\usepackage[
nonumberlist, %keine Seitenzahlen anzeigen
acronym,      %ein Abkürzungsverzeichnis erstellen
toc,          %Einträge im Inhaltsverzeichnis
section]      %optionen: [numberedsection|section] im Inhaltsverzeichnis auf "interpretiert" [chapter|section]-Ebene erscheinen
{glossaries}

% Ein eigenes Symbolverzeichnis erstellen
\newglossary[slg]{symbolslist}{syi}{syg}{Symbolverzeichnis}
 % Den Punkt am Ende jeder Beschreibung deaktivieren
\renewcommand*{\glspostdescription}{}
\makeglossaries
%~ \makeindex

\usepackage{bookmark}
%~ \hypersetup{%
	%~ draft,
	%pdfa,
	%unicode,
	%driverfallback=hpdftex,
	%hypertex,
	%colorlinks=false,pdfborderstyle={/S/U/W 1},% for non-PDF/A
	%~ colorlinks,
	%draft,%no links
	%~ allcolors=black,%for PDF/A with pdfx
%~ }
\usepackage{hyperref}
\usepackage[top=2cm,bottom=2.5cm,left=2.5cm,right=2.5cm]{geometry}

% minitoc
%~ \usepackage{minitoc}
% do not show minitoc error msgs
%~ \usepackage{silence}
%~ \WarningFilter{minitoc(hints)}{W0023}
%~ \WarningFilter{minitoc(hints)}{W0028}
%~ \WarningFilter{minitoc(hints)}{W0030}
%~ \WarningFilter{minitoc(hints)}{W0043}
%~ \WarningFilter{minitoc(hints)}{W0049}

% Kleiner Kommentar vor der bib, zitation funktioniert hierin nicht
\newcommand{\quellenHinweistext}{Online aufgefundene Quellen wurden via \href{https://archive.org/}{archive.org} gesichert und umfangreiche Werke bei Abgabe auf dem Datenträger beigelegt.}
