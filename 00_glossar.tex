%Glossar-Einträge
\newglossaryentry{glos:Dezentralisierung}{
name=Dezentralisierung,
description={Als Gegensatz zur Zentralisierung. Hier insb. bezüglich Teilnehmern in Computernetzwerken bzw. Entscheidungsprozessen. Die Steigerung ist: Zentralisiert, Dezentralisiert, Verteilt.}
}

\newglossaryentry{glos:Fork}{
name=Fork,
description={Aufspaltung in einer Blockchain, die zur zeitweisen oder dauerhaften Spaltung führen kann.%
}}

\newglossaryentry{glos:Hard Fork}{
name=Hard Fork,
description={Vereinfacht: Nicht rückwärtskompatible Protokolländerungen die ab einem bestimmten Block Gültigkeit erlangen.%
}}

\newglossaryentry{glos:Soft Fork}{
name=Soft Fork,
description={Vereinfacht: Rückwärtskompatible Protokolländerungen die ab einem bestimmten Block Gültigkeit erlangen.}
}

\newglossaryentry{glos:SmartContract}{
name=Smart Contract,
description={Auch Chaincode (Firma IBM) genannte Programme, die auf einer Blockchain bzw. im Netzwerk der Teilnehmer laufen. Die Umsetzung erfolgt über einen Stack-basierte Programmierung deren Ergebnis den State beeinflusst. Z.B. Bitcoin durch das an Assembler erinnernde \enquote{Script} und bei Ethereum davon abstrahiert über höhere Programmiersprachen.}
}

\newglossaryentry{glos:State}{
name=State,
description={Hier der Zustand bezüglich einer Blockchain oder eines Smart~Contract.}
}

\newglossaryentry{glos:Block}{
name=Block,
description={Bildlicher Begriff für eine Datenstruktur, die Transaktionen zusammenfasst.}
}

\newglossaryentry{glos:FinTech}{
name=Financial Services and Technology,
description={Unternehmen, die mit Unterstützung von moderner Technologie Finanzdienstleistungen anbieten% \autocite{w:lexika-econimics}
}}

\newglossaryentry{glos:CR}{
name=Zensurresistenz,
description={Wie stark ein betrachteter Aspekt gegen Verschweigen, Löschen oder Veränderung geschützt ist.}
}

\newglossaryentry{glos:Konsens}{
name=Konsens,
description={Engl. Consensus. Hier die als Wahrheit angenommene, algorithmisch gesteuerte Aussagen wie Reihenfolge aber auch Zeit oder State in einem verteilten Netzwerk.
}}

\newglossaryentry{glos:Mining}{
name=Mining,
description={Englischer Begriff für \emph{Schürfen}. Ein Verfahren zum finden des nächsten Blocks, insb. bei Blockchains die auf Proof of Work basieren. Die Tätigkeit der Netzwerkteilnehmer wird damit bildlich dem Freilegen von seltenen  gleichgesetzt.}
}

\newglossaryentry{glos:Transaktion}{
name=Transaktion,
description={Datenstruktur, die ein Asset }
}

\newglossaryentry{glos:IRC}{
name=Internet Relay Chat,
description={Chatsystem, textbasiert. Seit 1993 durch die IETF als ein Internet Standard mit informationellem Charakter angenommen. s.a. \url{https://tools.ietf.org/html/rfc1459}}
}

\newglossaryentry{glos:PKI}{
name=Public-Key-Infrastruktur,
description={Ein System, das die Verteilung und Überprüfung von öffentlichen kryptografischen Schlüsseln unterstützt. In Unternehmen werden hierzu häufig Zertifikate nach dem Standard X.509 der \href{https://www.itu.int/}{Internationalen Fernmeldeunion} bzw. auch ISO/IEC\,9594-8 verteilt.}
}

\newglossaryentry{glos:PoW}{
name=Proof of Work,
description={Ein auf Hashcash %(s. Whitepaper \cite{p:bitcoin})
basierendes Verfahren, dass die Schwierigkeit zur Erstellung neuer Blöcke durch stetiges Suchen nach einem Ziel steuert. Dieses Ziel kann u.a. durch die Anzahl der führenden Nullen %(s. \href{https://github.com/bitcoinbook/bitcoinbook/blob/8d01749bcf45f69f36cf23606bbbf3f0bd540db3/ch10.asciidoc\#proof-of-work-algorithm}{Proof of Work}% \autocite{b:mastering-bitcoin})
eines Hashes dargestellt werden, wie dies bei Bitcoin der Fall ist.}
}

\newglossaryentry{glos:PoS}{
name=Proof of Stake,
description={Ein Konsensalgorithmen mit Anreizsystem durch einen riskierten Einsatz des eigenen Vermögens (z.B. an Kryptowährung). Ein Problem ist die zufällige Auswahl der über die Richtigkeit bestimmenden Teilnehmer für den jeweiligen Block. Teilweise wird dieser Ansatz durch Delegation zu \emph{delegated Proof of Stake} (dPoS) ergänzt.}
}

\newglossaryentry{glos:Pruning}{
name=Pruning,
description={Engl. \emph{pruning}. Das Beschneiden/Zurechtstutzen, hier bezüglich der Einträge für Transaktionen in einem Block exkl. der Integritätsinformationen.%
}}

\newglossaryentry{glos:Schwierigkeit}{
name=Schwierigkeit,
description={Engl. \emph{difficulty} oder \emph{target}. Dient bei Konsens-Algorithmen für \gls{glos:PoW} zur Anpassung der Wahrscheinlichkeit des Entstehens eines neuen Blocks.
}}

\newglossaryentry{glos:Wallet}{
name=Wallet,
description={Entlehnt aus dem Engl. für Brieftasche als Bezeichnung für ein Programm, oder Datenstruktur bzw. Datei. Sie beinhaltet die kryptografischen Schlüssel für die Interaktion mit der Blockchain. 
}}

\newglossaryentry{glos:Ausgabeaufschlag}{
name=Ausgabeaufschlag,
description={Engl. blockreward. Die Belohnung -- insb. bei Bitcoin -- für den Erfolgsfall bei einem Proof-of-Konsensmechanismus (s.a. Halfing).
}}

\newglossaryentry{glos:Exchange}{
name=Exchange,
description={Engl. für Wechslestuben. Hier Dienstleister und Märkte die dem Tausch zwischen stattlichen Währungen und Kryptowährungen/Token dienen.%
}}

\newglossaryentry{glos:Halfing}{
name=Halfing,
description={Halbierung des Ausgabeaufschlags (engl. \emph{Block Reward}) zur Regulierung der Gesamtmenge.
Bei Bitcoin wird der Ausgabeaufschlag (neue Bitcoin) alle 210.000 Blocks (d.h. statistisch etwa alle 4 Jahre) halbiert. \\
50 Bitcoin betrug der initiale Ausgabeaufschlag im Januar 2009.\\
25 Bitcoin ab \href{https://blockchain.info/block/000000000000048b95347e83192f69cf0366076336c639f9b7228e9ba171342e}{Block 210.000} am 28.11.2012\\% \autocite{w:blockchain}.
12,5 Bitcoin ab \href{https://blockchain.info/block/000000000000000002cce816c0ab2c5c269cb081896b7dcb34b8422d6b74ffa1}{Block 420.000}) am 09.07.2016\\% \autocite{w:blockchain}.
Kommende Halbierungen werden über verschiedene Webseiten (u.a. \url{http://www.thehalvening.com/} o. \url{http://bitcoinclock.com/}) nach Auswertung der Entwicklung der Schwierigkeit im Netzwerk geschätzt. \\
Bei Ethereum ist die absolute Menge nicht final festgelegt (\url{https://ethereum.stackexchange.com/questions/443/what-is-the-total-supply-of-ether}), andere native Währungen (z.B. \href{http://dogecoin.com/}{Dogecoin} oder \href{http://www.monero.cc/}{Monero}) haben teilweise eine infinite Menge.
}}

\newglossaryentry{glos:Hashrate}{
name=Hashrate,
description={Die Anzahl der Hashes je Zeiteinheit (meist je Sekunde). Die Absolute Zahl der Hashes wird äquivalent zur verrichteten Arbeit verstanden. Einerseits eine Maßzahl für die erbrachte Leistung eines einzelnen Teilnehmers. Zumeist aber die aufgrund der Schwierigkeit geschätzte Gesamtleistung des Netzwerkes.}
}

\newglossaryentry{glos:GenesisBlock}{
name=Genesis Block,
description={In Anlehnung an die griechische Bedeutung Schöpfung/Geburt -- u.a. bekannt durch das 1. Buch Mose der Bibel -- der erste Block einer Blockchain. Dieser wird üblicherweise vom Initiator unabhängig vom Mining erstellt.}
}

\newglossaryentry{glos:TPS}{
name=Transaktionsgeschwindigkeit,
description={Metrik für die Anzahl je Zeiteinheit durchführbarer oder tatsächlich durchgeführten Transaktionen nach Konsens.}
}

\newglossaryentry{glos:LN}{
name=Lightning Network,
description={Ein Netzwerk für das Routing von Zahlungen zunächst im Bitcoin-Netzwerk und potentiell darüber hinaus in andere Blockchains.
Die Erweiterung erfolgt als zweite Protokollschicht über Limelocks und Multisignature-Adressen in einem Smart~Contract. Wiederverwendbare Payment~Codes die die Handhabung von Adressen erleichtern und gleichzeitig den Datenschutz verbessern werden integriert.}
}

%Abkürzungen
\newacronym{BTC}{BTC}{Bitcoin%
}
%Eine Abkürzung mit Glossareintrag
\newacronym{IRC}{IRC}{Internet Relay Chat\protect\glsadd{glos:IRC}
}
\newacronym{DNS}{DNS}{Domain Name System%\protect\glsadd{glos:AD}
}

\newacronym{DGN}{DGN}{Deutsches Gesundheitsnetz%\protect\glsadd{glos:AD}
}

\newacronym{DFN}{DFN}{Deutsches Forschungsnetz%\protect\glsadd{glos:AD}
}

\newacronym{PKI}{PKI}{Public-Key-Infrastruktur\protect\glsadd{glos:PKI}
}

\newacronym{LN}{LN}{Lightning Network\protect\glsadd{glos:LN}%
}

\newacronym{PoW}{PoW}{Proof of Work\protect\glsadd{glos:PoW}
}

\newacronym{PoS}{PoS}{Proof of Stake\protect\glsadd{glos:PoS}
}

\newacronym{TPS}{TPS}{Transaction Processing System\protect\glsadd{glos:TPS}%
}

\newacronym{GRNET}{GRNET}{Greek Research and Technology Network%
}

\newacronym{RPC}{RPC}{Remote Procedure Call%
}

\newacronym{JSON}{JSON}{JavaScript Object Notation%
}

\newacronym{MQTT}{MQTT}{Message Queuing Telemetry Transport%MQTT ist ein offenes Nachrichtenprotokoll für Machine-to-Machine-Kommunikation, das die Übertragung von Telemetriedaten in Form von Nachrichten zwischen Geräten ermöglicht, trotz hoher Verzögerungen oder beschränkten Netzwerken.
}

\newacronym{IoT}{IoT}{Internet of Things%
}

\newacronym{DLT}{DLT}{Distributed-Ledger-Technologie%
}

\newacronym{BC}{BC}{Blockchain%
}

\newacronym{BCT}{BCT}{Blockchain-Technologie%
}

\newacronym{DAG}{DAG}{Directed Acyclic Graph%
}

\newacronym{DApp}{DApp}{Distributed App(lication)%
}

\newacronym{MIT}{MIT}{Massachusetts Institute of Technology%
}

\newacronym{HSM}{HSM}{Hochschule Mittweida%
}

\newacronym{EEA}{EEA}{Enterprise Ethereum Alliance%
}

\newacronym{ANS}{ANS}{Antshares%
}

\newacronym{BCI}{BCI}{Blockchain-Implementierung%
}

\newacronym{HTW}{HTW}{Hochschule für Technik und Wirtschaft Dresden%
}

\newacronym{MOOC}{MOOC}{Massive Open Online Course%
}

\newacronym{IPDB}{IPDB}{Interplanetary Database%
}

\newacronym{IPFS}{IPFS}{Interplanetary Filesystem%
}

\newacronym{VDI}{VDI}{Verein Deutscher Ingenieure% (\href{https://www.vdi.de}{vdi.de})%
}

\newacronym{OB}{OB}{OpenBazaar%
}

\newacronym{ICO}{ICO}{Initial Coin Offering%
}

\newacronym{SE}{SE}{Sharing Economy%
}

\newacronym{IDE}{IDE}{Integrated Development Environment%
}

