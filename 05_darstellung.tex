%Methode: paarweiser Vergleich
%~ Um die in Frage stehenden Blockchain-Implementierungen miteinander vergleichen zu können,
%~ soll den eigenen Kriterien jeweils ein von der Software unabhängiges Gewicht als Präferenz für die Wichtigkeit zugeordnet werden.

%~ Spezifisch für jedes Kriterien 

Der Vergleich der gewonnen Kriterien soll nicht nur der Entscheidungsfindung dienen, sondern auch der Kommunikation der Entscheidung.
Nicht nur um sie nachvollziehbar zu machen, sondern auch weil im Zweifel mehrere Entscheider Einfluss nehmen möchten.
Daher steht die Frage der Darstellung hier im Vordergrund der Kommunikation.

%~ \footnote{ \autocite{p:vergleich}}


\section{Betrachtete Optionen zur Darstellung}

\subsection{Checklisten}

Eine sehr einfache Möglichkeit zur Erfolgskontrolle stellen Checklisten dar.
Sie kann zur überprüfung von binären Kriterien kann sehr einfach \enquote{abgehakt} werden.
 

\subsection{Ranking}

Die einfache Bewertung mit Attributen oder Zahlen kann zu einer Rangfolge bezüglich der Zielerfüllung gebracht werden.


\subsection{Entscheidungsquadrat}

%erschöpft in der Darstellung sich in 2-3 Dimensionen
%häufige Dimensionen Wert/Qualität

\subsection{Polarprofil}

\subsection{Paarweiser Vergleich}

\section{Anwendung auf die Kriterien}

\subsection{Quantifizierung und Klassifizierung}

\subsection{Gruppierung und Gewichtung}

Innerhalb der Gruppen können Einzelne Kriterien eine höhere Priorität haben als andere.
Um dies zu kennzeichnen wird empfohlen eine Gewichtung zuzuordnen.
%~ An dieser Stelle können die unterschiedlichen Wertebereiche auch gezielt mit einem weiteren Faktor ausgeglichen werden.

Die Kriterien wurden bereits in drei Gruppen zusammengefasst.
Diese Abstufung kann bei einer Iteration genutzt werden um 
%~ \begin{enumerate}
Einzelne Kriterien an- oder abzuwählen und
sich zügig auf die für eigene Zwecke notwendigen Gruppen zu konzentrieren.
%~ \end{enumerate}

